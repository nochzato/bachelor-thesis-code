%%% Основні відомості %%%
\newcommand{\UDC}                      % УДК
{(впишіть правильний УДК!)}            % УДК виглядає приблизно як 004.056.5 або 513.2, або навіть 004.056.5:513.2+519.1
% Для того, щоб знайти правильний УДК, використовуйте каталог https://teacode.com/online/udc/

\newcommand{\reportAuthor}             % ПІБ автора повністю
{Іванов Петро Сидорович}
\newcommand{\reportAuthorShort}        % ПІБ автора коротко
{Петро ІВАНОВ}
\newcommand{\reportAuthorGroup}        % група автора
{ФХ-N3}
\newcommand{\reportTitle}              % Назва роботи
{Назва дослідження дуже довга, не влізає в один рядочок аж ніяк взагалі ой людоньки що робити}
%% використовуйте символ "\par" або "\\" для розбиття назви на декілька рядків

\newcommand{\supervisorFio}            % Науковий керівник, ПІБ повністю
{Прізвище Ім'я По-батькові}
\newcommand{\supervisorFioShort}       % Науковий керівник, ПІБ коротко
{Ім'я ПРІЗВИЩЕ}
\newcommand{\supervisorRegalia}        % Науковий керівник: посада, степінь, звання
{посада, степінь, звання}              % наприклад: доцент кафедри ПЕКЛА, д.ф.-м.н., доцент
                                       % якщо виходить дуже довго - скорочуйте: доц. каф. ПЕКЛА, д.ф.-м.н., доц.

\newcommand{\consultFio}               % Консультант, ПІБ повністю
{}
\newcommand{\consultRegalia}           % Консультант: звання, степінь, посада
{}
% Якщо у вас нема консультанта - залишайте ці поля порожніми


\newcommand{\reviewerFio}              % Рецензент, ПІБ повністю
{Прізвище Ім'я По-батькові}                        
\newcommand{\reviewerRegalia}          % Рецензент: звання, степінь, посада
{посада, степінь, звання}

\newcommand{\YearOfDefence}            % рік захисту
{2023}
\newcommand{\YearOfBeginning}          % попередній рік - може, можна це якось автоматизувати, нє?
{2022}