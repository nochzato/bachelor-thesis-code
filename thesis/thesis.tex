% ********** Приклад оформлення пояснювальної записки **********
% *********  до атестаційної роботи ступеня бакалавра **********


\documentclass{bachelor_thesis}

% Додаткові пакети вносіть у цей файл
\input{01_packages}
\usepackage[style=numeric]{biblatex}
\addbibresource{thesis.bib}

% Додаткові визначення та перевизначення команд вносіть у цей файл
%%% У даному файлі визначайте всі необхідні вам нові команди TeX
%%% або робіть перевизначення існуючих, наприклад...

% Перевизначення символу порожньої множини та знаків "більше-дорівнює", "менше-дорівнює" на прийняті у нас
\let\oldemptyset\emptyset
\let\emptyset\varnothing
\let\geq\geqslant
\let\leq\leqslant

% Визначення нових математичних команд
\newcommand*{\binsp}[1]{\ensuremath \left\{0, 1\right\}^{#1}}       % {0, 1}^m
\newcommand*{\xor}{\ensuremath \oplus}                              % \xor = (+)
\newcommand*{\GF}[1]{\ensuremath \mathbb F_{#1}}                    % F_n
\newcommand*{\GFgroup}[1]{\ensuremath \mathbb F^{*}_{#1}}           % F^*_n
\newcommand*{\Zring}[1]{\ensuremath \mathbb Z_{#1}}                 % Z_n
\newcommand*{\Zgroup}[1]{\ensuremath \mathbb Z^{*}_{#1}}            % Z^*_n
\newcommand*{\Jset}[1]{\ensuremath \mathbb J_{#1}}                  % J_n
\newcommand*{\Qset}[1]{\ensuremath \mathbb Q_{#1}}                  % Q_n
\newcommand*{\PQset}[1]{\ensuremath \widetilde{\mathbb Q}_{#1}}     % Q~_n
\newcommand*{\cyclic}[1]{\ensuremath \left\langle {#1} \right\rangle}                  % <g>
\newcommand*{\Legendre}[2]{\ensuremath \left( \frac{#1}{#2} \right)}  % символ Лежандра/Якоби
\newcommand*{\compinv}[1]{\ensuremath {#1}^{\left\langle -1 \right\rangle}}  % обратный по композиции

% Інший спосіб визначення математичного оператору
\DeclareMathOperator{\ord}{ord}
\DeclareMathOperator{\lcm}{lcm}
\DeclareMathOperator{\Li}{Li}
\DeclareMathOperator{\Coef}{Coef}
\DeclareMathOperator{\Log}{Log}
\DeclareMathOperator{\Exp}{Exp}
\DeclareMathOperator{\Res}{Res}
\DeclareMathOperator{\charact}{char}
\DeclareMathOperator{\Sym}{Sym}


% команда для коментарів червоним кольором
% !!! Конфлікт пакету color з якимось іншим пакетом, не використовувати
%\newcommand{\todo}[1]{\textcolor{red}{#1}}


%%% ...і таке інше

% Бюрократичні відомості про автора роботи
%%% Основні відомості %%%
\newcommand{\UDC}                      % УДК
{(впишіть правильний УДК!)}            % УДК виглядає приблизно як 004.056.5 або 513.2, або навіть 004.056.5:513.2+519.1
% Для того, щоб знайти правильний УДК, використовуйте каталог https://teacode.com/online/udc/

\newcommand{\reportAuthor}             % ПІБ автора повністю
{Іванов Петро Сидорович}
\newcommand{\reportAuthorShort}        % ПІБ автора коротко
{Петро ІВАНОВ}
\newcommand{\reportAuthorGroup}        % група автора
{ФХ-N3}
\newcommand{\reportTitle}              % Назва роботи
{Назва дослідження дуже довга, не влізає в один рядочок аж ніяк взагалі ой людоньки що робити}
%% використовуйте символ "\par" або "\\" для розбиття назви на декілька рядків

\newcommand{\supervisorFio}            % Науковий керівник, ПІБ повністю
{Прізвище Ім'я По-батькові}
\newcommand{\supervisorFioShort}       % Науковий керівник, ПІБ коротко
{Ім'я ПРІЗВИЩЕ}
\newcommand{\supervisorRegalia}        % Науковий керівник: посада, степінь, звання
{посада, степінь, звання}              % наприклад: доцент кафедри ПЕКЛА, д.ф.-м.н., доцент
                                       % якщо виходить дуже довго - скорочуйте: доц. каф. ПЕКЛА, д.ф.-м.н., доц.

\newcommand{\consultFio}               % Консультант, ПІБ повністю
{}
\newcommand{\consultRegalia}           % Консультант: звання, степінь, посада
{}
% Якщо у вас нема консультанта - залишайте ці поля порожніми


\newcommand{\reviewerFio}              % Рецензент, ПІБ повністю
{Прізвище Ім'я По-батькові}                        
\newcommand{\reviewerRegalia}          % Рецензент: звання, степінь, посада
{посада, степінь, звання}

\newcommand{\YearOfDefence}            % рік захисту
{2023}
\newcommand{\YearOfBeginning}          % попередній рік - може, можна це якось автоматизувати, нє?
{2022}


% Починаємо верстку документа
\begin{document}

\pagestyle{plain}
\setfontsize{14}

% Створюємо титульну сторінку
% Титульный лист
\thispagestyle{empty}
\linespread{1.1}

\begin{center}
    {\bfseries
        НАЦІОНАЛЬНИЙ ТЕХНІЧНИЙ УНІВЕРСИТЕТ УКРАЇНИ \par
        <<КИЇВСЬКИЙ ПОЛІТЕХНІЧНИЙ ІНСТИТУТ \par
        імені Ігоря СІКОРСЬКОГО>>\par
        Навчально-науковий фізико-технічний інститут\par
        \medskip
        Кафедра математичного моделювання та аналізу даних}
\end{center}

\vspace{5mm}

\begin{tabularx}{\textwidth}{XX}
    <<На правах рукопису>> & <<До захисту допущено>>                                         \\[06pt]
    УДК \UDC               & Завідувач кафедри                                               \\[06pt]
                           & \rule{2.5cm}{0.25pt} Н. М. Куссуль                              \\[06pt]
                           & <<\rule{0.5cm}{0.25pt}>> \rule{2.5cm}{0.25pt} \YearOfDefence~р.
\end{tabularx}

%\linespread{1.5}                    % Неодинарный интервал
\begin{center}
    \vspace{5mm}
    {\bfseries\huge Дипломна робота} \par
    {\bfseries на здобуття ступеня бакалавра} \par
\end{center}

зі спеціальності: 113 Прикладна математика \par
на тему: \textbf{<<\reportTitle>>}

\vspace{5mm}

\begin{tabularx}{\textwidth}{>{\setlength\hsize{1.5\hsize}}X >{\setlength\hsize{0.5\hsize}}X}
    Виконав:                                  &                      \\
    студент 4 курсу, групи \reportAuthorGroup &                      \\
    \reportAuthor                             & \rule{2.5cm}{0.25pt} \\[12pt]
    Керівник:                                 &                      \\
    \supervisorRegalia                        &                      \\
    \supervisorFio                            & \rule{2.5cm}{0.25pt} \\[12pt]
    %%%%% Якщо у вас зненацька є консультант у роботі - розкоментуйте наступні три рядки (а цей - не розкоментовуйте!)
    %Консультант:                                             & \\
    %\consultRegalia                                          & \\
    %\consultFio                                              & \rule{2.5cm}{0.25pt}   \\[12pt]
    Рецензент:                                &                      \\
    \reviewerRegalia                          &                      \\
    \reviewerFio                              & \rule{2.5cm}{0.25pt}
\end{tabularx}

\vspace{15mm}

\linespread{1.1}                    % Майже одинарный интервал
\begin{tabularx}{\textwidth}{>{\setlength\hsize{1.25\hsize}}X >{\setlength\hsize{1.5\hsize}}X >{\setlength\hsize{0.25\hsize}}X}
     & Засвідчую, що у цій дипломній роботі немає запозичень з праць інших
    авторів без відповідних посилань.

     &                                                                       \\
     & Студент \rule{2.5cm}{0.25pt}                                        &
\end{tabularx}

%\vspace{10mm}
\vfill
\begin{center}
    {Київ~---~\YearOfDefence}
\end{center}

\newpage
\thispagestyle{plain}

% Створюємо завдання
% Титульный лист
\linespread{1.1}

\begin{center}
{\bfseries
НАЦІОНАЛЬНИЙ ТЕХНІЧНИЙ УНІВЕРСИТЕТ УКРАЇНИ \par
<<КИЇВСЬКИЙ ПОЛІТЕХНІЧНИЙ ІНСТИТУТ \par
імені Ігоря СІКОРСЬКОГО>>\par
Навчально-науковий фізико-технічний інститут\par
Кафедра математичних методів захисту інформації}
\end{center}
\par

\linespread{1.1}
Рівень вищої освіти --- перший (бакалаврський)

Спеціальність --- 113~Прикладна математика,

ОПП <<Математичні методи криптографічного захисту інформації>>

\vspace{10mm}
\begin{tabularx}{\textwidth}{XX}
& ЗАТВЕРДЖУЮ                              \\[06pt]
& В.о. завідувача кафедри                 \\[06pt]
& \rule{2.5cm}{0.25pt} Сергій ЯКОВЛЄВ     \\[06pt]
& <<\rule{0.5cm}{0.25pt}>> \rule{2.5cm}{0.25pt} \YearOfDefence~р. 
\end{tabularx}

\vspace{5mm}
\begin{center}
{\bfseries ЗАВДАННЯ \par}
{\bfseries на дипломну роботу \par}
\end{center}

%%%%%====================================
% !!! Не чіпайте наступні три команди!
%%%%%====================================
\frenchspacing
\doublespacing          % інтервал "1,5" між рядками, тепер навічно
\setfontsize{14}

Студент: \reportAuthor \par

1. Тема роботи: <<\emph{\reportTitle}>>,
науковий керівник дисертації: \supervisorRegalia ~\supervisorFio, \par
затверджені наказом по університету \No \rule{0.5cm}{0.25pt} від <<\rule{0.5cm}{0.25pt}>> \rule{2.5cm}{0.25pt} \YearOfDefence~р.

2. Термін подання студентом роботи: <<\rule{0.5cm}{0.25pt}>> \rule{2.5cm}{0.25pt} \YearOfDefence~р.

% Коли будете заповнювати пункти 3-10, приберіть команду \emph --- вона тільки для виділення моїх коментарів
3. Об'єкт дослідження: \emph{(впишіть об'єкт дослідження)}

4. Предмет дослідження: \emph{(впишіть предмет дослідження)}

5. Перелік завдань: \emph{(впишіть теми та задачі, які ви розкриваєте у роботі; можна робити це попунктно)}

6. Орієнтовний перелік графічного (ілюстративного) матеріалу: \emph{(якщо у вас є окремий ілюстративний матеріал окрім власне роботи (креслення, макети тощо), зазначайте; інакше вказуйте <<Презентація доповіді>>)}

7. Орієнтовний перелік публікацій: \emph{(впишіть наявні публікації або <<планується доповідь на всеукраїнській конференції>>)}

8. Дата видачі завдання: 10 вересня \YearOfBeginning~р.

% Якщо перша частина завдання вилізе за сторінку - приберіть команду \newpage
% Календарний план є продовженням завдання, а не окремою частиною

\newpage

\begin{center}
Календарний план
\end{center}

\renewcommand{\arraystretch}{1.5}
\begin{table}[h!]
\setfontsize{14pt}
\centering
    \begin{tabularx}{\textwidth}{|>{\centering\arraybackslash\setlength\hsize{0.25\hsize}}X|>{\setlength\hsize{2\hsize}}X|>{\centering\arraybackslash\setlength\hsize{1\hsize}}X|>{\centering\arraybackslash\setlength\hsize{0.75\hsize}}X|}
    \hline \No\par з/п & Назва етапів виконання магістерської дисертації & Термін виконання & Примітка \\
    \hline 
    % номер етапу
    1 & 
    % назва етапу
    Узгодження теми роботи із науковим керівником & 
    % термін виконання
    01-15 вересня \YearOfBeginning~р. &
    % примітка - зазвичай "Виконано"
    Виконано \\
%%% -- початок інтервалу для копіювання
    \hline 
    % номер етапу
    2 & 
    % назва етапу
    Огляд опублікованих джерел за тематикою дослідження & 
    % термін виконання
    Вересень-жовтень \YearOfBeginning~р. &
    % примітка - зазвичай "Виконано"
    Виконано \\
%%% -- кінець інтервалу для копіювання
% не прибирайте амперсанди та \\ наприкінці рядків!
% скопійовані інтервали вставляти перед фінальною \hline та заповнювати відповідно
% ось так:
%%% -- початок інтервалу для копіювання
    \hline 
    % номер етапу
    3 & 
    % назва етапу
    \ldots & 
    % термін виконання
    \ldots &
    % примітка - зазвичай "Виконано"
    Виконано \\
%%% -- кінець інтервалу для копіювання
    \hline %фінальна hline
    \end{tabularx}
\end{table}

\renewcommand{\arraystretch}{1}
\begin{tabularx}{\textwidth}{>{\setlength\hsize{1.2\hsize}}X >{\setlength\hsize{0.5\hsize}}X >{\setlength\hsize{1.3\hsize}}X}
Студент  & \rule{2.5cm}{0.25pt}  & \reportAuthorShort \\[06pt]
Керівник & \rule{2.5cm}{0.25pt}  & \supervisorFioShort \\
\end{tabularx}

\newpage


% У даному костильному рішенні перші три сторінки (титул та завдання на 
% роботу) друкуються окремо від основної частини тез.
% Тому перша сторінка сформованого документу нумерується як четверта

% Створюємо анотації
%\setcounter{page}{4}
%!TEX root = ../abstract.tex

\abstractUkr

Кваліфікаційна робота містить: ??? стор., ??? рисунки, ??? таблиць, ??? джерел.

У рефераті роботи ви повинні коротко (два-три абзаци) викласти, що саме 
було зроблено у цій роботі. Перші три речення реферату (після статистичних 
даних) повинні окреслити мету роботи, об'єкт та предмет дослідження. Після 
цього викладаються основні результати, одержані в ході дослідження.

Наприкінці анотації великими літерами зазначаються ключові слова. Ось так:

% наприкінці анотації потрібно зазначити ключові слова
\MakeUppercase{КЛЮЧОВІ СЛОВА, СИМЕТРИЧНА КРИПТОГРАФІЯ, ФІЗТЕХ НАЙКРАЩІЙ}


%%%% Рішенням кафедри з 2018 року ми прибираємо анотації російською мовою
% \abstractRus
%
%Русская аннотация должна быть точным переводом украинской (включая 
%статистику и ключевые слова).

\abstractEng

The English abstract must be the exact translation of the Ukrainian 
``annotation'' (including statistical data and keywords).

% Не прибирайте даний рядок
\clearpage

% Створюємо зміст
%\pagenumbering{gobble}
\tableofcontents
\cleardoublepage
%\pagenumbering{arabic}
%\setcounter{page}{8}    %!!! -- продумати, як автоматизувати номер сторінки

% Створюємо перелік умовних позначень, скорочень і термінів
% Якщо цей розділ вам не потрібен, просто закоментуйте два наступних рядка
\shortings
%!TEX root = ../thesis.tex
% створюємо перелік умовних позначень, скорочень і термінів
ФТІ --- Фізико-технічний інститут

$\xor$ --- операція побітового додавання  %зауважте, що використовується перевизначена операція \xor


(Якщо ви не використовуєте перелік умовних позначень, просто приберіть 
даний розділ.)

(БУДЬ ЛАСКА, ПРОСЛІДКУЙТЕ, ЩОБ НОМЕР СТОРІНКИ СПІВПАДАВ ІЗ СПРАВЖНІМ! Це залежить від того, наскільки великим є ваш зміст.
Номер сторінки проставляється у файлі thesis.tex, рядок 35.)

% Створюємо вступ
\intro
%!TEX root = ../thesis.tex
% створюємо вступ
\textbf{Актуальність дослідження.} Актуальність даного дослідження полягає 
у тому, що без нього ви не одержите диплом про вищу освіту. Відповідно, ви повинні 
оформити результати вашого дослідження належним чином.

Вступ є однією із самих формалізованих частин дипломної роботи. На початку 
ви у двох-трьох абзацах повинні окреслити проблематику та актуальність 
вашого дослідження, після чого переходити до мети та завдання.

\textbf{Метою дослідження} є певна абстрактна недосяжна річ на кшталт 
загальнолюдського щастя на горизонті. Для досягнення мети необхідно 
розв'язати \textbf{задачу дослідження}, яка полягає у чомусь суттєво більш 
конкретному. Для розв'язання задачі необхідно вирішити такі завдання:

\begin{enumerate}
\item провести огляд опублікованих джерел за тематикою дослідження;
\item (наступний пункт, пов'язаний із теоретичним дослідженням);
\item (і ще один, наприклад, про експериментальну перевірку результатів);
\item (і взагалі, краще із науковим керівником проконсультуйтесь, як ваші 
завдання правильно писати).
\end{enumerate}

\emph{Об'єктом дослідження} є якісь процеси або явища загального 
характеру (наприклад, <<інформаційні процеси в системах криптографічного 
захисту>>).

\emph{Предметом дослідження} є конкретний математичний чи фізичний 
об'єкт, який розглядається у вашій роботі та який можна трактувати
як певну властивість об'єкта дослідження (наприклад, <<моделі та методи
диференціального криптоаналізу ітеративних симетричних блочних шифрів>>).

При розв’язанні поставлених завдань використовувались такі \emph{методи дослідження}: і 
тут коротенький перелік (наприклад, але не обмежуючись: методи лінійної та абстрактної 
алгебри, теорії імовірностей, математичної статистики, комбінаторного 
аналізу, теорії кодування, теорії складності алгоритмів, методи 
комп’ютерного та статистичного моделювання) 

\textbf{Наукова новизна} отриманих результатів полягає... -- тут необхідно 
перелічити, що саме нового з точки зору науки несе ваша робота. До усіх 
тверджень, які сюди виносяться, подумки (а іноді й явним чином) потрібно 
ставити слово <<вперше>> -- і ці твердження повинні залишатись істинними.

\textbf{Практичне значення} результатів полягає... -- тут необхідно 
зазначити практичну користь від результатів вашої роботи. Що саме можна 
покращити, підвищити (або знизити), зробити гарного (або уникнути 
поганого) після вашого дослідження.

\textbf{Апробація результатів та публікації.} Наприкінці вступу необхідно 
зазначити перелік конференцій, семінарів та публікацій, в яких викладено 
результати вашої роботи. Якщо результати вашої роботи ніде не 
доповідались, опускайте даний абзац.

% Додаємо глави
% Якщо ваша робота містить менше або більше глав - модифікуйте наступні 
% рядки відповідним чином
%!TEX root = ../thesis.tex

\chapter{(Назва першого розділу)}
\label{chap:review}  %% відмічайте кожен розділ певною міткою -- на неї наприкінці необхідно посилатись

На початку кожного розділу рекомендується вставити одне-два-абзац речень, 
у яких коротенько представили, про що тут взагалі буде мова.

\section{(Назва першого підрозділу)}

Перший розділ повинен бути присвячений огляду попередніх результатів за 
тематикою вашого дослідження. У даному розділі повинні міститись вс' 
визначення та описи, необхідні для подальшого викладення матеріалу, та результати 
ваших попередників.

Зауважимо, що наводити детальні доведення не ваших результатів необхідно 
наводити лише тоді, коли вони містять якусь вкрай важливу інформацію для 
саме ваших результатів.

Також зауважимо, що абсолютно на всі не ваші результати повинні стояти 
належним чином оформлені посилання.

Розмір першого (оглядового) розділу не повинен перевищувати третини вашої 
дипломної роботи (без урахування додатків).


\section{(Назва другого підрозділу)}

Наведемо основні правила оформлення текстів у системі \LaTeX.

Для абзацу робіть пусті рядки у файлі. Курсивний текст робиться командою 
emph: \emph{ось так}. Жирний текст робиться командою textbf: \textbf{ось так}.

<<Лапки>> робляться двома знаками більше та двома знаками менше. Довге 
тире у тексті --- трьома дефісами, коротке -- двома дефісами; у формулах 
мінуси робляться одним дефісом: $a-b$.

Пишіть звичайний текст звичайним текстом, а формули, позначення змінних та 
операцій (усі формули, усі позначення змінних та усі операції) беріть у 
знаки долара, ось так: $E = mc^2$, $a_1 = a^{(2)} \cdot a_{n, k}$, $e^x = 
\sum_{k = 0}^{\infty} {\frac{x^k}{k!}}$. Якщо вам 
не подобається, як \LaTeX подав формулу для експоненти (мені, наприклад, 
не подобається), то можна внести у код формули деякі корективи та написати ось так: $e^x 
= \sum\limits_{k = 0}^{\infty} {\dfrac{x^k}{k!}}$.

Для прикладу різні варіації коми у формулах: $(a, b)$ vs. $(a,b)$. Поки 
пакет icomma працює, різниця видна наочно.

Виключна формула (формула окремим рядком) робиться через подвійні знаки 
долара або через оточення equation. Зауважте, що при цьому змінюється 
оформлення формул:
$$e^x = \sum_{k = 0}^{\infty} {\frac{x^k}{k!}}.$$

Формули за помовчанням не підтримують кирилічні літери. Зверніть увагу на 
порожній рядок перед попереднім реченням у tex-файлі: без нього не буде 
створено абзац.

Із більш специфічних позначень --- ось так, скажімо, можна подати 
перестановку:
$$\pi = \begin{pmatrix}
1 & 2 & 3 & 4 & 5 & 6 & 7 & 8 & 9\\
a & 5 & 9 & 6 & 4 & 8 & 2 & 1 & 7
\end{pmatrix},$$
де $a=3$. Зауважте, що у попередньому реченні нема порожнього рядочку 
перед <<де>> (та, відповідно, абзацу після формули), а кома внесена у 
виключну формулу, бо інакше вона переїде у наступний рядок тексту.

Декілька формул поспіль треба збирати в єдине ціле оточеннями gather або 
eqnarray; назви оточень із зірочками вказують \LaTeX'у не нумерувати дані 
формули. Наприклад, ось рекуренти для циклових чисел та чисел Стірлінга 
I~роду:
\begin{eqnarray*}
c(n+1, k) &=& c(n, k-1)+nc(n, k); \\
s(n+1, k) &=& s(n, k-1)-ns(n, k).
\end{eqnarray*}

Зверніть увагу на символ <<\verb|~|>> у попередньому абзаці tex-файлу між 
<<I>> та <<роду>>; це нерозривний пробіл, який не дасть рознести пов'язані 
частини по різних рядках. Тільду треба ставити перед усіма посиланнями 
(команди ref та cite), перед тире та у місцях, які не можна розривати за 
правилами граматики.

Для специфічних позначень ви можете задавати власні команди (їх 
рекомендовано заносити у файл <<\verb|02_redefinitions|>>). Наприклад, 
подивіться, як оформлюється теорема Лагранжа-Бюрмана із використанням 
введених команд \verb|\Coef| та \verb|\compinv|:

\begin{theorem}[Лагранж, Бюрман] \label{thLagrangeBurmann}
Для будь-якого ряду $A \in x \mathcal R[[x]]_1$ та $k \in \mathbb N$ справедливе співвідношення
$$n \Coef[x^n] \left( \compinv{A}(x) \right)^k = k \Coef[x^{n-k}] \left(\! \frac{x}{A(x)} \!\right)^n.$$
\end{theorem}
\begin{proof}
Доведення ви подивитесь деінде, а тут подивіться, як воно оформлюється 
(зокрема, на квадратик наприкінці :)).                                                                                       
\end{proof}

\begin{corollary} \par %\label{pr##}
Будь-ласка, перевіряйте граматику. Латеховські редактори зазвичай не мають 
інтегрованих спелчекерів української мови, тому використовуйте сервіси, 
наведені, наприклад, тут: https://coma.in.ua/30584
\end{corollary}

Іноді написаний файл треба компілювати двічі для одержання ефекту 
(скажімо, для коректної побудови усіх гіперпосилань та побудови змісту). 
Скажімо, оце посилання на теорему~\ref{thLagrangeBurmann} (теорему 
Лагранжа-Бюрмана) з першої компіляції може показати вам знаки питання 
<<??>>. Однак після повторної компіляції ви одержите те, що потрібно.

Онлайн-сервіси на кшталт Overleaf справляються з такими ситуаціями за одну компіляцію. Однак той 
же Overleaf має звичку компілювати pdf-файли навіть за наявності помилок у 
тексті, просто ігноруючи відповідні місця. Якщо ви працюєте у Overleaf, 
то переконайтесь, що у вас нема червоних помилок після компіляції. На 
щастя, останні апдейти Overleaf вивалюють червоні помилки прямо вам в очі, 
тому їх нескладно помітити.

Якщо вам потрібна якась фіча, запитайте в Сенсея. Майже напевно вона є.

На жаль деякі пакети шаблону викликають незрозумілі конфлікти. Поки що не 
вдалось інтегрувати у шаблон такі пакети, як color та tikz. Якщо без 
кольорового забарвлення тексту ще можна пережити, то використовувати 
діаграми tikz поки що рекомендується за допомогою милиць:

\begin{itemize}
\item створюєте окремий допоміжний tex-проект, у якому за допомогою tikz створюєте 
діаграму;
\item компілюєте допоміжний tex-проект;
\item вставляєте створену діаграму з pdf-файлу у диплом як зображення.
\end{itemize}

Ми ж зі свого боку продовжуємо працювати над покращенням даного шаблону.

\section{(Назва третього підрозділу)}


Надамо деякі рекомендації щодо використання даного стильового файлу.

\begin{theorem}
Використовуйте оточення \emph{theorem} для теорем.
\end{theorem}
\begin{proof}
Для доведень використовуйте оточення \emph{proof}.
\end{proof}
\begin{theorem}
Нумерація відбувається автоматично
\end{theorem}
\begin{claim}
Використовуйте оточення \emph{claim} для тверджень.
\end{claim}
\begin{lemma}
Використовуйте оточення \emph{lemma} для лем.
\end{lemma}
\begin{corollary}
Використовуйте оточення \emph{corollary} для наслідків.
\end{corollary}
\begin{definition}
Використовуйте оточення \emph{definition} для визначень.
\end{definition}
\begin{example}
Використовуйте оточення \emph{example} для прикладів, на які є посилання.
\end{example}
\begin{remark}
Використовуйте оточення \emph{remark} для зауважень. Зверніть увагу, як 
веде себе команда \textbf{emph}
\end{remark}


\chapconclude{\ref{chap:review}}

Наприкінці кожного розділу ви повинні навести коротенькі підсумки по його 
результатах. Зокрема, для оглядового розділу в якості висновків необхідно 
зазначити, які задачі у даній тематиці вже були розв'язані, а саме 
поставлена вами задача розв'язана не була (або розв'язана погано), тому у 
наступних розділах ви її й розв'язуєте.

Якщо ваш звіт складається з одного розділу, пропускайте висновок до 
нього~-- він повністю включається в загальні висновки до роботи
%!TEX root = ../thesis.tex
% створюємо розділ
\chapter{(Назва другого розділу)}
\label{chap:theory}

До другого розділу також краще написати малесенький вступ. Зокрема, це
збільшує загальний об'єм роботи та покращує її читабельність.

\section{(Якийсь підрозділ)}

У другому розділі необхідно наводити розв'язання поставленої перед вами
задачі у теоретичному або аналітичному сенсі (хоча, звісно, все залежить
від того, яка саме задача перед вами поставлена).

Для подання матеріалів можна використовувати таблиці (наприклад,
Таблицю \ref{tab_weight}). Розмір шрифту у таблиці може бути меншим за 14~pt (наприклад, 12~pt, або навіть 10~pt, якщо так таблиця виглядає зрозуміліше та компактніше).

\begin{table}[ht]
    \setfontsize{14pt}
    \caption{Розрахунок якоїсь фантастичної дичини у декілька кроків}
    \label{tab_weight}
    \centering
    \begin{tabular}{|c|c|c|c|c|c|c|c|c|}
        \hline \multirow{2}{*}{Параметр $x_i$} & \multicolumn{4}{c|}{Параметр $x_j$} &
        \multicolumn{2}{c|}{Перший крок}       & \multicolumn{2}{c|}{Другий крок}                                                                     \\
        \cline{2-9}                            & $X_1$                               & $X_2$            & $X_3$ & $X_4$ & $w_i$ &
        ${K_\text{в}}_i$                       & $w_i$                               & ${K_\text{в}}_i$                                               \\
        \hline $X_1$                           & 1                                   & 1                & 1.5   & 1.5   & 5     & 0.31 & 19    & 0.32 \\
        \hline $X_2$                           & 1                                   & 1                & 1.5   & 1.5   & 5     & 0.31 & 19    & 0.32 \\
        \hline $X_3$                           & 0.5                                 & 0.5              & 1     & 0.5   & 2.5   & 0.16 & 9.25  & 0.16 \\
        \hline $X_4$                           & 0.5                                 & 0.5              & 1.5   & 1     & 3.5   & 0.22 & 12.25 & 0.20 \\
        \hline \multicolumn{5}{|c|}{Разом:}    & 16                                  & 1                & 59.5  & 1                                   \\
        \hline
    \end{tabular}
\end{table}

Бажано, щоб кожен пункт завдань, окреслених у вступі, відповідав певному
розділу або підрозділу у дипломній роботі.

\begin{theorem}
    Нумерація у наступних розділах також проставляється автоматично та коректно.
\end{theorem}

\section{(Якийсь наступний підрозділ з дуже-дуже довгою назвою, загальна кількість слів в якій, однак, не повинна перевищувати 12 слів)}

Для подання матеріалів також дуже зручними є рисунки (наприклад, рисунки
\ref{fig_sudak} або \ref{fig_pacman}).

\begin{figure}[ht]
    \centering
    \begin{subfigure}[b]{0.5\textwidth}
        \includegraphics[scale=0.3]{Images/Sudak.png}
        \caption{}
        % обратите внимание на знак % после \end{subfigure} и 
        % отсутствие пустых строк и разделителей после \end{subfigure}
        % -- это сливает в одну строку подфигуры
    \end{subfigure}%
    \begin{subfigure}[b]{0.5\textwidth}
        \includegraphics[scale=0.3]{Images/Tudak.png}
        \caption{}
    \end{subfigure}

    \caption{Різні види риб: (a) судак, (б) тудак.}
    \label{fig_sudak}
\end{figure}

\begin{figure}[ht]
    \centering
    \includegraphics[scale=0.5]{Images/Pacman.jpg}
    \caption{Частка кругових діаграм, які схожі на Пекмена}
    \label{fig_pacman}
\end{figure}

\section{Оформлення посилань}

Посилання до роботи рекомендовано робити за допомогою підсистеми BibTex.
Це максимально зручно, повірте мені. Усі притомні світові журнали надають
посилання на свої статті у вигляді bib-файлів, з яких просто треба
перенести відомості у свій файл. Існують інструменти автоматизованого створення bib-файлів, наприклад, JabRef.
Якщо у вас нема автоматизованого механізму, то посилання на джерела у
bib-файлах легко створюються вручну. Головне --- не забувати чотири простих
правила:

1) Імена авторів необхідно подавати у вигляді <<Прізвище, Ім'я>> та
розділяти ключовим словом <<and>>, наприклад:

\texttt{author = "Яковлєв, Сергій Володимирович and Дамблдор, Альбус Персиваль Вульфрік Брайан"}

2) Після усіх полів запису необхідно ставити кому, окрім останнього.

3) Джерела у переліку посилань з'являться тільки після того, як ви зробите
посилання на нього у тексті, наприклад, див. роботи.

\chapconclude{\ref{chap:theory}}

Наприкінці розділу знову наводяться коротенькі підсумки.
%!TEX root = ../thesis.tex
\chapter{(Назва третього розділу)}
\label{chap:practice}

\section{(якийсь підрозділ)}

Подивіться, як нераціонально використовується простір, якщо не писати 
вступи до розділів. :)

Зазвичай третій розділ присвячено опису практичного застосування або 
експериментальної перевірки аналітичних результатів, одержаних у другому 
розділі роботи. Втім, це не обов'язкова вимога, і структура основної 
частини диплому більш суттєво залежить від характеру поставлених завдань. 
Навіть якщо у вас є певне експериментальне дослідження, але його загальний 
опис займає дві сторінки, то краще приєднайте його підроздіром у 
попередній розділ.

При описі експериментальних досліджень необхідно:

\begin{itemize}
\item наводити повний опис експериментів, які проводились, параметрів 
обчислювальних середовищ, засобів програмування тощо;
\item наводити повний перелік одержаних результатів у чисельному вигляді для їх можливої 
перевірки іншими особами;
\item представляти одержані результати у вигляді таблиць та графіків, 
зрозумілих людському оку;
\item інтерпретувати одержані результати з точки зору поставленої задачі 
та загальної проблематики ваших досліджень.
\end{itemize}

У жодному разі не потрібно вставляти у даний розділ тексти 
інструментальних програм та засобів (окрім того рідкісного випадку, коли 
саме тексти програм і є результатом проведення експериментів). За 
необхідності тексти програм наводяться у додатках.


\chapconclude{\ref{chap:practice}}

Висновки до останнього розділу є, фактично, підсумковими під усім 
дослідженням; однак вони повинні стостуватись саме того, що розглядалось у 
розділі.


% Створюємо висновки
\conclusions
%!TEX root = ../thesis.tex
% створюємо Висновки до всієї роботи
Загальні висновки до роботи повинні підсумовувати усі ваші досягнення у 
даному напрямку досліджень.

За кожним пунктом завдань, поставлених у вступі, у висновках повинен 
міститись звіт про виконання: виконано, не виконано, виконано частково (І 
чому саме так). Наприклад, якщо першим поставленим завданням у вас іде 
<<огляд літератури за тематикою досліджень>>, то на початку висновків ви 
повинні зазначити, що <<у ході даної роботи був проведений аналіз 
опублікованих джерел за тематикою (...), який показав, що (...)>>. Окрім 
простої констатації про виконання ви повинні навести, які саме результати 
ви одержали та проінтерпретувати їх з точки зору поставленої задачі, мети 
та загальної проблематики.

В ідеалі загальні висновки повинні збиратись з висновків до кожного 
розділу, але ідеал недосяжний. :) Однак висновки не повинні містити 
формул, таблиць та рисунків. Дозволяється (та навіть вітається) 
використовувати числа (на кшталт <<розроблена методика дозволяє підвищити 
ефективність пустопорожньої балаканини на $2.71\%$>>).

Наприкінці висновків необхідно зазначити напрямки подальших досліджень: 
куди саме, як вам вважається, необхідно прямувати наступним дослідникам у 
даній тематиці.


% Додаємо бібліографію
%%%%%%% Якщо ви не використовуєте bibtex, оце все треба закоментувати звідси...
\defbibenvironment{bibliography}
  {\list
     {\printfield[labelnumberwidth]{labelnumber}}
     {\setfontsize{14}%
     }}
  {\endlist}
  {\item}

\printbibliography[heading=bibintoc,title={Перелік посилань}]
%%%%%%% ...по осюди

% Якщо у вас виникли проблеми з бібтехом (у мене, наприклад, виникли), то замість бібтеху
% можна використати менш приємний, але завжди працюючий спосіб: вписати та оформити бібліографію вручну
% Для цього закоментуйте команду \printbibliography та розкоментуйте наступний рядок
% VVV оцей рядок розкоментуйте VVV
%%!TEX root = ../thesis.tex
% створюємо список використаної літератури
\begin{thebibliography}    
    \bibitem{sad} 
    asda

    \bibitem{dca}
    asd [Електронний ресурс]. --- Режим доступу: \url{dsf}.

 
\end{thebibliography}

% ^^^ оцей-оцей ^^^
% Бібліографія вноситься у файл "w2_bibliography.tex"

%%% А це взагалі перестало працювати :(
%\bibliographystyle{ugost2008}
%\bibliography{thesis}


% Створюємо додатки (дивись у файли додатків для необхідних пояснень)
% Якщо ви маєте меншу або більшу кількість додатків, модифікуйте наступні 
% рядки відповідним чином
% Якщо ви не маєте додатків, просто закоментуйте наступні рядки
\input{Chapters/z1_appendix_A}
\input{Chapters/z2_appendix_B}


% Нарешті
\end{document}