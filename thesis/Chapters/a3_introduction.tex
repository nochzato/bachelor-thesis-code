%!TEX root = ../thesis.tex
% створюємо вступ
\textbf{Актуальність дослідження.} Повномасштабне вторгнення Російської Федерацію
в Україну призвело до значних ушкоджень сільськогосподарських угідь.
Деформація ґрунтового покриву внаслідок формування фортифікаційних
споруд (окопів), утворення кратерів від бомб, хімічне забруднення внаслідок
використання боєприпасів –- це лише короткий перелік наслідків вторгнення~\cite{golubtsov2023}.
Задача даної робити -- проаналізувати існуючі методи розв'язання задачі класифікації
на супутникових зображеннях і виявлення найбільше ефективного.
Використання результатів поставлених задач роботи можуть бути застосовані для оцінки пошкоджень,
визначення найбільш вразливих зон для населення, доказів воєнних злочинів Російської Федерації, тощо.

\textbf{Метою дослідження} є вивчення і дослідження методів розпізнавання кратерів від бомбардувань.
Для досягнення мети необхідно розв'язати наступні \textbf{задачі дослідження}:

\begin{enumerate}
    \item провести огляд опублікованих джерел за тематикою дослідження;
    \item (наступний пункт, пов'язаний із теоретичним дослідженням);
    \item (і ще один, наприклад, про експериментальну перевірку результатів);
    \item (і взагалі, краще із науковим керівником проконсультуйтесь, як ваші
          завдання правильно писати).
\end{enumerate}

\emph{Об'єктом дослідження} є процес виявлення кратерів за супутниковими знімками.

\emph{Предметом дослідження} методи виявлення кратерів, сформованих внаслідок вибухів
бомб чи інших вибухових пристроїв, з використанням сучасних технологій обробки зображень,
супутникового зондування та геоінформаційних систем.

При розв’язанні поставлених завдань використовувались такі \emph{методи дослідження}: і
тут коротенький перелік (наприклад, але не обмежуючись: методи лінійної та абстрактної
алгебри, теорії імовірностей, математичної статистики, комбінаторного
аналізу, теорії кодування, теорії складності алгоритмів, методи
комп’ютерного та статистичного моделювання)

\textbf{Наукова новизна} отриманих результатів полягає у проведенні порівнняня між існуючими
методами розв'язання задачі.

\textbf{Практичне значення} результатів полягає у тому,
що робота відкриває перспективи для покращення ефективності військових
та гуманітарних досліджень.

\textbf{Апробація результатів та публікації.} Наприкінці вступу необхідно
зазначити перелік конференцій, семінарів та публікацій, в яких викладено
результати вашої роботи. Якщо результати вашої роботи ніде не
доповідались, опускайте даний абзац.